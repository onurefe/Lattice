\documentclass{article}
\usepackage[utf8]{inputenc}
\usepackage{amsmath}

\title{Piezoelectric Control}
\author{Onur Efe}
\date{July 2020}

\begin{document}

\maketitle

\section{Obtaining linear phase control measure}
Impedance of the series rlc element can be given as;

\begin{multline}
    Z = r + jwl + \frac{1}{jwc}
    = \frac{jwrc - w^2lc + 1}{jwc} \\
\end{multline}

Since multiplicative inverse of the impedance $Z$ is proportional to complex power, real power should be proportional to the real part of $\frac{1}{Z}$. According to this normalized complex power $p$ is introduced as;

\begin{multline}
    p = \frac{1}{Z} = \frac{jwc}{jwrc - w^2lc + 1} \\
\end{multline}

Normalized real power can be calculated as;
\begin{multline}
    p_{r} = \Re{\left[\frac{jwc}{(1-w^2lc) + jwrc} \right]}
    = \Re{\left[\frac{jwc((1-w^2lc) - jwrc)}{(1-w^2lc)^2 + (wrc)^2} \right]} \\
    = \Re{\left[\frac{((jwc-jw^3lc^2) - j^2w^2rc^2)}{(1-w^2lc)^2 + (wrc)^2} \right]}
    = \frac{w^2rc^2}{(1-w^2lc)^2 + (wrc)^2} \\
    = \frac{w^2rc^2}{w^4l^2c^2 - 2w^2lc + w^2r^2c^2 + 1} = \frac{w^2rc^2}{w^4l^2c^2 + w^2(r^2c^2 - 2lc) + 1}
\end{multline}

Expression isn't suitable for linearization. But $\frac{p_{img}}{p_r}$ can be a much more suitable measure;
\begin{multline}{\label{temp:1}}
    \frac{p_{img}}{p_r} 
    = \frac{\frac{wc-w^3lc^2}{(1-w^2lc)^2 + (wrc)^2} - \frac{1}{wc_p}}{\frac{w^2rc^2}{(1-w^2lc)^2 + (wrc)^2}}
    = \frac{wc-w^3lc^2}{w^2rc^2} - \frac{{(1-w^2lc)^2 + (wrc)^2}}{w^3 rc^2 c_p} \\
    = \frac{1-w^2lc}{wrc} - \frac{{(1-w^2lc)^2 + (wrc)^2}}{w^3 rc^2 c_p}
\end{multline}

Adding the normalized reactive power consumed by capacitance;
\begin{multline}
    \frac{1-w^2lc - \frac{1}{wc_p}}{wrc} 
\end{multline}

Introducing quality factor which is defined defined as;
\begin{equation}{\label{def:qualityfactor}}
    q = \frac{1}{r} \sqrt{\frac{l}{c}}
\end{equation}

And $w_0$ is defined as;
\begin{equation}{\label{def:omegazero}}
    w_0 = \frac{1}{\sqrt{lc}}
\end{equation}

Now variations of the load can be obtained from the variations
of the resistance. Since the piezoelectric elements are pretty stable
we can assume that $l, c$ doesn't change with loading; only $r$
changes. Variables with subscript $0$, $1$ denoting values before and after
loading respectively;

\begin{equation}
    \frac{q_1}{q_0} =
    \frac{\frac{1}{r_1} \sqrt{\frac{l}{c}}}{\frac{1}{r_0} \sqrt{\frac{l}{c}}}
    = \frac{\frac{1}{r_1}}{\frac{1}{r_0}}
    = \frac{r_0}{r_1}
\end{equation}

So when loaded, since the quality factor decreases, resistance will increase
proportionally. So, in general, measuring complex impedance would give the resistance
value as well; so data wheter the system is loaded or not can be obtained in very high
rates.

Advantage of this method is there aren't any dependence of any other process and
control doesn't interact with other controls.

\section{Obtaining parallel capacitance}

\begin{multline}
    I = I_{cap} + I_{rlc} = \frac{V}{jwc_p} + \frac{jVwc}{jwrc - w^2lc +1} \\
    \frac{V(jwrc - w^2lc +1)}{(jwrc - w^2lc +1) . (jwc_p)} + \frac{(jVwc) . (jwc_p)}{(jwc_p).(jwrc - w^2lc +1)} \\
    = \frac{-Vw^2cc_p + V(jwrc - w^2lc +1)}{-w^2rcc_p - jw^3lcc_p +jwc_p}
    = \frac{V(jwrc - w^2lc + 1 - w^2cc_p)}{-w^2rcc_p - jw^3lcc_p +jwc_p} \\
    \implies Z = \frac{-w^2rcc_p - jw^3lcc_p +jwc_p}{jwrc - w^2lc + 1 - w^2cc_p}
\end{multline}

Another way of obtaining the equivalent circuit model could be to use Taylor approximation applied to the
impedance polynomial.

\section{Power control with varying duty cycle}
Components of the Fourier series of a pulse wave;

\begin{equation}{\label{eq:fourierseries}}
    a_n = \frac{2A}{n\pi} \sin{(n \pi d)}
\end{equation}

$V, d$ denoting the amplitude of the pulse wave applied and duty cycle respectively, average power transmitted to the piezo device(considering that the first harmonic does match with the resonance frequency of the piezo device) does equal to;

\begin{multline}{\label{temp:2}}
    P_{drv} = \frac{a_1^2}{2r} = \frac{4A^2}{1^2 \pi^2 2 r} \sin^2{(\pi d)} = \frac{2A^2}{\pi^2 r} \frac{1 - cos(2 \pi d)}{2} = \frac{A^2}{\pi^2 r} (1 - cos(2 \pi d)) \\
    \implies 1 - \frac{P_{drv} \pi^2 r}{A^2} = cos(2 \pi d) \implies d = \frac{1}{2\pi} cos^{-1}(1 - \frac{P_{drv} \pi^2 r}{A^2})
\end{multline}

$P_{drvmax} = \frac{2A^2}{\pi^2 r}$. Now introducing normalized power $P_{norm} = \frac{P_{drv}}{P_{drvmax}}$ we can express \eqref{temp:2} as;

\begin{equation}
    d = \frac{1}{2\pi} cos^{-1}(1 - 2P_{norm})
\end{equation}

Now introducing an approximation for $cos^{-1}(x)$;
\begin{equation}
    cos^{-1}(x) = \frac{\pi}{2} + \frac{x(a + bx^2)}{1 + x^2(c + dx^2)}
\end{equation}

where;
\begin{multline}
    \begin{bmatrix}
        a \\
        b \\
        c \\
        d
    \end{bmatrix}
    =
    \begin{bmatrix}
        -0.939115566365855  \\
        0.9217841528914573  \\
        -1.2845906244690837 \\
        0.295624144969963174
    \end{bmatrix}
    \\
\end{multline}

\section{Resonance Frequency Tracking Algorithm}
Secant method like method will be used for resonance frequency tracking. Measure $m = \frac{p_{img}}{p_r}$ will be tried to make zero.
$m_i, f_i$ denoting $i.th$ measure and the frequency at iteration $i$, $f_{i+1}$ is;

\begin{equation}
    f_{i+1} = f_i - \frac{m_i(f_i - f_{i-1})}{m_i - m_{i-1}}
\end{equation}
There should be some error about frequency shifting.

\end{document}

